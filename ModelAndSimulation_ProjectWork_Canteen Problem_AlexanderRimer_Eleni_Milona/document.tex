\documentclass[12pt,english,titlepage,a4paper,twoside,bibliography=totoc]{scrreprt}   
%BCOR8.25mm seht hier bei f�r den abstand von der linken seite, da diese
% gebunden wird
%%%%%%%%%%%%%%%%%%%%%%%%%%%%%%%%%%%%%%%%%%%%%%%%%%%%%%%%%%%%%%%%%%%%%%%%%
%EINGEBUNDENE PAKETE (immer vor dem document ) 
%%%%%%%%%%%%%%%%%%%%%%%%%%%%%%%%%%%%%%%%%%%%%%%%%%%%%%%%%%%%%%%%%%%%%%%%%
\usepackage{bibgerm}			%fuer aufwendigeres literaturverzeichnis mit themes
%\usepackage{amssymb}			% wichtige Symbole
\usepackage{graphics}			% Einbinden von Graphiken
\usepackage[pdftex]{graphicx}
\usepackage{epstopdf}			%automatischen generieren von pdfs aus eps
\usepackage{array} 
\usepackage[centertags]{amsmath}  
\usepackage{amsfonts}			% Schriftart
\usepackage{a4}				% Seitenformat
\usepackage{epsfig}				% Zusaetzliche Graphikbefehle 
\usepackage{floatflt}			% Tabellen
\usepackage{scrpage2}			% useheadings
\usepackage[latin1]{inputenc}	% Umlaute in Windows 
\usepackage{paralist}			% Bessere Enumerate-Umgebungen (Aufzaehlungszeichen
% koennen gewaehlt werden) \usepackage{tabular}			% Tabellen
\usepackage{booktabs}
\usepackage[]{babel}	
\usepackage{mparhack}		% bessere Randnotizen
\usepackage{setspace}		% eigentlich nur fuer den Titel ..
\usepackage{multirow}		% Row-/Columnspan ..
\usepackage{longtable}		%fuer mehrseitige tabellen
\usepackage{makeidx}		%hiermit kann man ein index erstellen, z.b. fuer
% abkuerngen und sozu
\usepackage[T1]{fontenc}	%erweiterter buchstabensatz mit umlauten 
\usepackage{url}			%um URLs darzustellen
\usepackage{thmbox}			%umgebung fuer theoreme und beispiele
\usepackage{listings}		%umgebung f�r programmcode
\usepackage{color}
\usepackage{xcolor}
\usepackage{caption}
\usepackage{calc}
\usepackage{floatflt}
\usepackage{float}
% \setlength{\textwidth}{16.5cm}
% \setlength{\evensidemargin}{-0.5cm}
% \setlength{\oddsidemargin}{0.5cm}
% \setlength{\marginparwidth}{18mm}
% \reversemarginpar
% \let\oldmarginpar\marginpar 
% \renewcommand\marginpar[1]{\-\oldmarginpar[\raggedleft\scriptsize #1]%
% {\raggedright\scriptsize #1}}

 %%%%%%%%%%%%%%%%%%%%%%%%%%%%%%%%%%%%%%%%%%%%%%%%%%%%%%%%%%%%%%%%%%%%%%%%%
%FORMATIERUNG DES LISTINGS 
%%%%%%%%%%%%%%%%%%%%%%%%%%%%%%%%%%%%%%%%%%%%%%%%%%%%%%%%%%%%%%%%%%%%%%%%%
\DeclareCaptionFont{white}{\color{white}}
\DeclareCaptionFormat{listing}{\colorbox{gray}{\parbox{\textwidth-2\fboxsep}{#1#2#3}}}
\captionsetup[lstlisting]{format=listing,labelfont=white,textfont=white}
%SMC-Sprache 
\lstdefinelanguage{smc}{ 
	keywords={Entry, Exit,class,fsmclass,package,start,map,include,import,String,push,pop},
	morecomment=[s]{/*}{*/},
	backgroundcolor=\color{light_grey_costum},
	basicstyle=\footnotesize\ttfamily,           % the size of the fonts that are used for the code
	numbers=left,                   % where to put the line-numbers 
  	numberstyle=\tiny\color{gray},  % the style that is used for the line-numbers
  	stepnumber=1,                   % the step between two line-numbers. If it's 1, each line 
                                  % will be numbered
    numbersep=5pt,                  % how far the line-numbers are from the code
        % choose the background color. You must add \usepackage{color}
    showspaces=false,               % show spaces adding particular underscores
    showstringspaces=false,         % underline spaces within strings
    showtabs=false,                 % show tabs within strings adding particular underscores
    frame=lines,                   	  % values: none, leftline, topline,
							      %bottomline, lines (top and bottom), single for single frames, or shadowbox.
    rulecolor=\color{black},        % if not set, the frame-color may be changed on line-breaks within not-black text (e.g.
   % commens
  % (green here))
    tabsize=2,                      % sets default tabsize to 2 spaces
    captionpos=b,                   % sets the caption-position to bottom
    breaklines=true,                % sets automatic line breaking
%      xleftmargin=1.25em,
%    	framexleftmargin=0.5em 
}
 

%SMC-Farben
\definecolor{commentGray}{RGB}{99,99,99}
\definecolor{transitionColor}{RGB}{179,22,51}
\definecolor{stateColor}{RGB}{39,68,46}
\definecolor{keywordColor}{RGB}{21,25,181}
\definecolor{light_grey_costum}{RGB}{229,229,229}

%ACCELEO-Sprache
\lstdefinelanguage{acceleo}{
	keywords={module, import,file,if,else,query,public,include,import,for,let,endif,and,not,then,elseif,or,template},
	extendedchars=true,
	morecomment=[s]{'}{'},
	backgroundcolor=\color{light_grey_costum},
	basicstyle=\footnotesize\ttfamily,           % the size of the fonts that are used for the code
	numbers=left,                   % where to put the line-numbers 
  	numberstyle=\tiny\color{gray},  % the style that is used for the line-numbers
  	stepnumber=1,                   % the step between two line-numbers. If it's 1, each line 
                                  % will be numbered
    numbersep=5pt,                  % how far the line-numbers are from the code
        % choose the background color. You must add \usepackage{color}
    showspaces=false,               % show spaces adding particular underscores
    showstringspaces=false,         % underline spaces within strings
    showtabs=false,                 % show tabs within strings adding particular underscores
    frame=lines,                   	  % values: none, leftline, topline,
							      %bottomline, lines (top and bottom), single for single frames, or shadowbox.
    rulecolor=\color{black},        % if not set, the frame-color may be changed on line-breaks within not-black text (e.g.
   % commens
  % (green here))
    tabsize=2,                      % sets default tabsize to 2 spaces
    captionpos=b,                   % sets the caption-position to bottom
    breaklines=true,                % sets automatic line breaking
%      xleftmargin=1.25em,
%    	framexleftmargin=0.5em 
}

%SMC-Farben
\definecolor{query}{RGB}{127,0,85}
\definecolor{templateColor}{RGB}{195,5,5}
\definecolor{queryAufruf}{RGB}{80,80,255}
\definecolor{keywordColor}{RGB}{21,25,181}
\definecolor{stringColor}{RGB}{63,127,127}
\definecolor{niceGreenColor}{RGB}{0,127,0}

\DeclareCaptionFormat{figure}{\colorbox{gray}{\parbox{\textwidth-2\fboxsep}{#1#2#3}}}
\captionsetup[figure]{format=figure,labelfont=white,textfont=white}


 

%%%%%%%%%%%%%%%%%%%%%%%%%%%%%%%%%%%%%%%%%%%%%%%%%%%%%%%%%%%%%%%%%%%%%%%%%
%tABELLENFORMATIERUNG
%%%%%%%%%%%%%%%%%%%%%%%%%%%%%%%%%%%%%%%%%%%%%%%%%%%%%%%%%%%%%%%%%%%%%%%%%
%  neuer  Befehl:  \includegraphicstotab[..]{..}
%  Verwendung  analog  wie  \includegraphics
\newlength{\myx}  %  Variable  zum  Speichern  der  Bildbreite
\newlength{\myy}  %  Variable  zum  Speichern  der  Bildh�he
\newcommand\includegraphicsToTab[2][\relax]{%
%  Abspeichern  der  Bildabmessungen
\settowidth{\myx}{\includegraphics[{#1}]{#2}}%
\settoheight{\myy}{\includegraphics[{#1}]{#2}}%
%  das  eigentliche  Einf�gen
\parbox[c][1.1\myy][l]{\myx}{%
\includegraphics[{#1}]{#2}}%
}%  Ende  neuer  Befehl


%%%%%%%%%%%%%%%%%%%%%%%%%%%%%%%%%%%%%%%%%%%%%%%%%%%%%%%%%%%%%%%%%%%%%%%%%
%METAINFORMATIONEN
%%%%%%%%%%%%%%%%%%%%%%%%%%%%%%%%%%%%%%%%%%%%%%%%%%%%%%%%%%%%%%%%%%%%%%%%%
\usepackage[
	pdftitle={Projectwork: University of Bayreuth: Canteen problem},
	pdfsubject={University of Bayreuth: Canteen problem},
	pdfauthor={Alexander Rimer, Eleni Milano},
	pdfkeywords={Modeling and Simulation, Event-Based-Modeling},
	colorlinks=true,
    linkcolor=black,
    citecolor=black,
    filecolor=black, 
    pagecolor=black,
    urlcolor=black]{hyperref} %Metainformationen  

%%%%%%%%%%%%%%%%%%%%%%%%%%%%%%%%%%%%%%%%%%%%%%%%%%%%%%%%%%%%%%%%%%%%%%%%%
%SCHRIFT FUER CAPTIONS AENDERN
%%%%%%%%%%%%%%%%%%%%%%%%%%%%%%%%%%%%%%%%%%%%%%%%%%%%%%%%%%%%%%%%%%%%%%%%%
\setkomafont{caption}{\footnotesize \selectfont}
\setkomafont{captionlabel}{\footnotesize \bfseries}
%%%%%%%%%%%%%%%%%%%%%%%%%%%%%%%%%%%%%%%%%%%%%%%%%%%%%%%%%%%%%%%%%%%%%%%%% 
%BEGINN
%%%%%%%%%%%%%%%%%%%%%%%%%%%%%%%%%%%%%%%%%%%%%%%%%%%%%%%%%%%%%%%%%%%%%%%%%
\begin{document}
%%%%%%%%%%%%%%%%%%%%%%%%%%%%%%%%%%%%%%%%%%%%%%%%%%%%%%%%%%%%%%%%%%%%%%%%%
%		Definitionen					%
%%%%%%%%%%%%%%%%%%%%%%%%%%%%%%%%%%%%%%%%%%%%%%%%%%%%%%%%%%%%%%%%%%%%%%%%%
\newcommand{\Titel}{University of Bayreuth: Canteen problem}
\newcommand{\Autor}{Alexander Rimer, Eleni Milano}
\newcommand{\ErsterPruefer}{Prof. Dr. Torsten Eymann}
%\newcommand{\ZweiterPruefer}{Prof. Dr. Ing. Stefan Jablonski}
\newcommand{\Betreuer}{Gaurang Phadke}
\newcommand{\Einreichungsdatum}{31. January 2012}%\today
\newtheorem[S,bodystyle=\normalfont]{Bsp}{Beispiel}[chapter]%definiere eigene
% umgebung


%%%%%%%%%%%%%%%%%%%%%%%%%%%%%%%%%%%%%%%%%%%%%%%%%%%%%%%%%%%%%%%%%%%%%%%%%
%		Titelseite Bachelorarbeit				%
%%%%%%%%%%%%%%%%%%%%%%%%%%%%%%%%%%%%%%%%%%%%%%%%%%%%%%%%%%%%%%%%%%%%%%%%%
\thispagestyle{empty}
% \setcounter{chapter}{0} %setzen der counter
% \setcounter{section}{0}
 
\begin{titlepage}
	\begin{figure}[!ht] 
	\end{figure} 
	\begin{center}
		\vspace*{20mm} 
		\textbf{\Huge{\Titel}} \\[25mm] 
		{\large Project Work} \\[5mm]
		{\large von} \\[5mm]
		{\large \Autor} \\[25mm]
		\textsc{\Large Lehrstuhl f\"ur Wirtschaftsinformatik} \\[3mm]
		\textsc{\Large Angewandte Informatik XX} \\[3mm]
  		\textsc{\Large University of  Bayreuth} \\[30mm]
		\vspace*{2cm}
		\begin{tabular}{p{8cm} l}   
		
		& \textbf{Pr"ufer:} \\
		Filling date:\hspace{0.6cm} \Einreichungsdatum  
		& \ErsterPruefer\\ 
 	%	& \ZweiterPruefer\\
 		& \textbf{Betreuer:} \\ 
 		& \Betreuer  		 
 	\end{tabular}
	\end{center}
\end{titlepage}
\thispagestyle{empty}
\setcounter{chapter}{0} %setzen der counter
\setcounter{section}{0}
\ \newpage


%%%%%%%%%%%%%%%%%%%%%%%%%%%%%%%%%%%%%%%%%%%%%%%%%%%%%%%%%%%%%%%%%%%%%%%%%%
%ZUSAMMENFASSUNGEN
%%%%%%%%%%%%%%%%%%%%%%%%%%%%%%%%%%%%%%%%%%%%%%%%%%%%%%%%%%%%%%%%%%%%%%%%%%
\renewcommand{\baselinestretch}{1.1}\normalsize

\chapter*{Abstract}% english version
\linespread{1.2} 
\thispagestyle{empty}
  -Problem der zu langen Warteschlagen in der Kantine von Bayreuth soll modelliert werden\\
  -Modellierungs soll m�glichst genau sein \\
  -\textbf{Es wird ein "`Event basierter"' Modellierungsansatzt verwendet}\\
  -Ziel ist es m�gliche Engp�sse herauszuarbeiten und diese gegebenfalls �ber die Variations von Parametern zu l�sen\\
  -Auch soll versucht werden ein bestm�gliches Nutzen/Leistungs Ansatz zu realisieren\\
  -->sodass man evtl. Optimierungspotentiale erkennt und diese in der Modellierung umsetzt (min. bzw. max. Anzahl an Angestellten and er Kasse)\\
  \cite{Acceleo:2012}

%%%%%%%%%%%%%%%%%%%%%%%%%%%%%%%%%%%%%%%%%%%%%%%%%%%%%%%%%%%%%%%%%%%%%%%%%%
%INHALTSVERZEICHNIS
%%%%%%%%%%%%%%%%%%%%%%%%%%%%%%%%%%%%%%%%%%%%%%%%%%%%%%%%%%%%%%%%%%%%%%%%%%
 
\addtocontents{toc}{\protect\thispagestyle{empty}} %unterdrueckt die seitenzahl
% bei aktueller seite fuer das verzeichnis 
\include{chap_2/introduction}

\include{chap_3/literal_review}
\include{chap_4/statement_of_regularities} 
\include{chap_5/description_of_the_model}
\include{chap_6/description_of_the_parameters}
\include{chap_7/description_of_the_results}
\include{chap_8/discussion_of_steps}
\include{chap_9/conclusion}
\include{chap_9_10/acknowledgments}

\tableofcontents 
\thispagestyle{empty}
 
%%%%%%%%%%%%%%%%%%%%%%%%%%%%%%%%%%%%%%%%%%%%%%%%%%%%%%%%%%%%%%%%%%%%%%%%%% 
%GLOSSAR drucken
%%%%%%%%%%%%%%%%%%%%%%%%%%%%%%%%%%%%%%%%%%%%%%%%%%%%%%%%%%%%%%%%%%%%%%%%%%
% \chapter*{\underline{Abk"urzungen}} 
% \thispagestyle{empty} %keine seitenzahl festlegen
% \begin{tabular}{p{0.1\textwidth}p{0.01\textwidth}p{0.5\textwidth}}
% 
% \end{tabular}
%%%%%%%%%%%%%%%%%%%%%%%%%%%%%%%%%%%%%%%%%%%%%%%%%%%%%%%%%%%%%%%%%%%%%%%%%%
%INHALT
%%%%%%%%%%%%%%%%%%%%%%%%%%%%%%%%%%%%%%%%%%%%%%%%%%%%%%%%%%%%%%%%%%%%%%%%%%
% \setcounter{chapter}{0} %setzen der counter
% \setcounter{section}{0}
\setcounter{page}{0}
 
 
%%%%%%%%%%%%%%%%%%%%%%%%%%%%%%%%%%%%%%%%%%%%%%%%%%%%%%%%%%%%%%%%%%%%%%%%%%
%ABBILDUNGSVERZEICHNIS
%%%%%%%%%%%%%%%%%%%%%%%%%%%%%%%%%%%%%%%%%%%%%%%%%%%%%%%%%%%%%%%%%%%%%%%%%%
\addcontentsline{toc}{chapter}{\listfigurename} %erzwinige dass das ins
% verzeichnis kommt
\listoffigures
% \thispagestyle{empty} 
%%%%%%%%%%%%%%%%%%%%%%%%%%%%%%%%%%%%%%%%%%%%%%%%%%%%%%%%%%%%%%%%%%%%%%%%%%
%CODEBEISPIEL-VERZEICHNIS
%%%%%%%%%%%%%%%%%%%%%%%%%%%%%%%%%%%%%%%%%%%%%%%%%%%%%%%%%%%%%%%%%%%%%%%%%%
%\addcontentsline{toc}{chapter}{Listings} 
\lstlistoflistings
%%%%%%%%%%%%%%%%%%%%%%%%%%%%%%%%%%%%%%%%%%%%%%%%%%%%%%%%%%%%%%%%%%%%%%%%%%
%LITERATUR
%%%%%%%%%%%%%%%%%%%%%%%%%%%%%%%%%%%%%%%%%%%%%%%%%%%%%%%%%%%%%%%%%%%%%%%%%%
\bibliographystyle{geralpha}
%\addcontentsline{toc}{chapter}{\bibname}
\bibliography{literature/literature}
%%%%%%%%%%%%%%%%%%%%%%%%%%%%%%%%%%%%%%%%%%%%%%%%%%%%%%%%%%%%%%%%%%%%%%%%%%
%ERKLAERUNG
%%%%%%%%%%%%%%%%%%%%%%%%%%%%%%%%%%%%%%%%%%%%%%%%%%%%%%%%%%%%%%%%%%%%%%%%%%

\end{document}